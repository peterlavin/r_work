% This file was converted to LaTeX by Writer2LaTeX ver. 0.4b
% see http://www.hj-gym.dk/~hj/writer2latex for more info
\documentclass[12pt,twoside]{article}
\usepackage[ascii]{inputenc}
\usepackage[T1]{fontenc}
\usepackage[english]{babel}
\usepackage{amsmath,amssymb,amsfonts,textcomp}
\usepackage{color}
\usepackage{calc}
\usepackage{longtable}
\usepackage{hyperref}
\hypersetup{colorlinks=true, linkcolor=blue, filecolor=blue, pagecolor=blue, urlcolor=blue}
\newcommand\textsubscript[1]{\ensuremath{{}_{\text{#1}}}}
% Outline numbering
\setcounter{secnumdepth}{0}
% List styles
\newcommand\liststyleLi{%
\renewcommand\labelitemi{{\textbullet}}
\renewcommand\labelitemii{{\textbullet}}
\renewcommand\labelitemiii{{\textbullet}}
\renewcommand\labelitemiv{{\textbullet}}
}
% Pages styles (master pages)
\makeatletter
\newcommand\ps@HTML{%
\renewcommand\@oddhead{}%
\renewcommand\@evenhead{}%
\renewcommand\@oddfoot{}%
\renewcommand\@evenfoot{}%
\setlength\paperwidth{8.5in}\setlength\paperheight{11in}\setlength\voffset{-1in}\setlength\hoffset{-1in}\setlength\topmargin{0.3937in}\setlength\headheight{12pt}\setlength\headsep{0cm}\setlength\footskip{12pt+0cm}\setlength\textheight{11in-0.3937in-0.3937in-0cm-12pt-0cm-12pt}\setlength\oddsidemargin{0.7874in}\setlength\textwidth{8.5in-0.7874in-0.3937in}
\renewcommand\thepage{\arabic{page}}
\setlength{\skip\footins}{0.0398in}\renewcommand\footnoterule{\vspace*{-0.0071in}\noindent\textcolor{black}{\rule{0.25\columnwidth}{0.0071in}}\vspace*{0.0398in}}
}
\makeatother
\pagestyle{HTML}
\setlength\tabcolsep{1mm}
\renewcommand\arraystretch{1.3}
\title{ST7016{}-statistics{}-report}
\begin{document}
\clearpage\pagestyle{HTML}
{\centering
\textbf{\textup{ST7016 Group Project}} 
\par}

{\centering
Group 1 
\par}

{\centering
Peter Lavin\newline
Ronan Watson\newline
Kathryn Cassidy\newline
Theresa Doyle\newline
Arifur Rahman 
\par}


\bigskip

{\centering

\par}

\begin{center}
[Warning: Image not found]
\end{center}
{\centering
\textbf{Analysis of user data from an eLearning application} 
\par}


\bigskip


\bigskip


\bigskip

\textbf{Table of Contents} 
Chapter 1: Introduction 

Origins of the data

Data Collection

Population and Sampling

Variables measured

System log data

Survey Responses

Other data

Questions and Analyses

Chapter 2: Summary Statistics

Chapter 3: Graphs and Tables

Chapter 4: Normality of the Data

Chapter 5: Relationships and Correlations

Chapter 6: Time Series Analysis

Chapter 7: Other...?

Chapter 8: Conclusions

Appendix A: The data file

Appendix B: the time series data file

\bigskip

\section[Chapter 1: Introduction ]{Chapter 1: Introduction }
\subsection[Origins of the data ]{\textrm{Origins of the data} }
Our data comes from the {\textquotedbl}eLGrid{\textquotedbl} eLearning
application which aims to teach learners about Grid technologies. A
brief overview of the eLGrid system will help to understand some of the
variables measured. 

The system uses adaptive eLearning technology to provide a personalised
navigation menu to learners. Each concept in the navigation menu is
annotated with a coloured {\textquotedbl}traffic{}-light{\textquotedbl}
icon. The colour chosen indicates the status of the materials
associated with that concept: 

\liststyleLi
\begin{itemize}
\item Green means that the concept is ready for viewing 
\item Amber means that the learner has viewed the material for this
concept and there is now a test available. It reminds the learner to
click on the {\textquotedbl}Test Me{\textquotedbl} menu item and
complete the test for that concept 
\item Red indicates that the concept is not ready for viewing as the
learner has not yet viewed or completed tests in certain prerequisite
concepts 
\item Clear indicates that the learner has
{\textquotedbl}learned{\textquotedbl} the concept, i.e. they have
correctly answered the test questions associated with this concept 
\item Grey indicates that the concept is not required material for that
learner (e.g. the same course might cater for biologists who want to
use Grid technologies and to Computer Scientists who need to know how
the underlying Grid technologies work, some concepts will be common to
both types of learners, but others will be relevant only to one or the
other. 
\end{itemize}
Some concepts do not have any tests associated with them; in which case
the traffic{}-light icon goes directly from green to clear when the
learner views the material, bypassing amber. 

~ 

\textit{Figure 1: The eLGrid eLearning system showing the personalised
navigation menu} 

\begin{center}
[Warning: Image not found]
\end{center}
\newline


~ 

\subsection[Data Collection ]{\textrm{Data Collection} }
Learner activity is logged automatically by the system, giving us a
number of metrics such as number of logins, number of times course
material is viewed, number of tests taken and passed, etc. There is
also timing data showing how long learners have spent viewing course
materials which we consider as a continuous variable. 

Some of the learners completed an end{}-of{}-course survey which gives
information about their opinion of, and satisfaction with, the system. 

A smaller subset also have a set of final scores. These were
undergraduates who were given access to the system in order to learn a
particular technology which they would use for their practical
exercises. The students were given a mark based on their answers to the
eLGrid tests along with a mark for their practical work; and these are
combined into a total mark. Most users, however, are research students
and scientists attending short workshops on Grid technologies, and for
these users there is no final score to capture. 

\subsection[Population and Sampling ]{\textrm{Population and Sampling} }
The data analysed here is a sample of the population of actual and
possible Grid users and we wish to use our analysis to make inferences
about this population in order to design better eLearning tools for
Grid users, and to be in a position to encourage new users to adopt
Grid technologies. 

We are analysing the data from all learners who have used the eLGrid
system, however as experimenters we have little control over the makeup
of this group. For example, it was not possible to design the sampling
methods in order to generate a representative sample. Instead the
makeup of the sample depends very much on which users chose to sign up
to use the eLGrid system or to attend workshops and courses where the
system was used. For the purposes of our analysis we assume that the
sample available to us is a representative sample of the population of
Grid users. 

\subsection[Variables measured ]{\textrm{Variables measured} }
In this section we will explain each of the variables in our data,
identifying their type and any other salient points. 

\subsubsection[System log data]{\rmfamily\bfseries System log data}
\textbf{learnerid} is an arbitrary identifier assigned by the system and
thus no analysis is required. 

\textbf{totalconcepttime} is the measure of the total time each learner
spent viewing course materials. It is generated by summing the times
measured from the learner clicks on on a particular concept to the time
that he/she clicks on any other link. In some cases the learner does
not click on another link in that session, e.g. the learner might close
their browser without clicking on the logout button. In this case a
default time of 30 seconds is assumed. The time is stored in seconds,
not milliseconds or any smaller units, but we will nonetheless treat it
as a continuous quantitative variable. 

\textbf{total\_concept\_count} is the total number of course concepts
which each learner has viewed. It is a discrete quantitative variable. 

\textbf{complete\_initial\_questionaire\_count} gives the number of
times the learner completed an initial questionnaire. This initial
questionnaire just asks the learner whether he/she is from a Computer
Science or an Application Science (e.g. biology, physics, medical
science, etc.) background. It is optional so many learners may not have
completed it at all and they can complete it multiple times, perhaps
changing their choice of background each time. It is a discrete
quantitative variable. 

\textbf{correct\_answer\_count} tells us the number of times that the
learner correctly answered the test questions for a concept. Note that
if a concept has multiple associated test questions, then the learner
must answer all of them correctly in one go before this will be counted
as a correct answer. The variable is a count and is thus a discrete
quantitative variable. 

\textbf{incomplete\_answer\_count} is the number of incomplete answers
to test questions. An incomplete answer occurs where there are multiple
questions for a concept and the learner only answers some of them, or
answers some correctly and others incorrectly (recall that all answers
for a given concept must be correctly answered in one go in order for
an attempt to be considered a correct answer).
Incomplete\_answer\_count is a discrete quantitative variable. 

\textbf{incorrect\_answer\_count} gives the number of incorrect answers
to test questions; again, this means that all questions associated with
a particular concept were answered incorrectly. If the learner got one
answer wrong and one right then that would be considered an incomplete
answer. This is a discrete quantitative variable. 

\textbf{login\_count} is the number of times the learner logged into the
system. It is a discrete quantitative variable. 

\textbf{logout\_count} is the number of times the learner logged out.
Often learners just close down the browser without logging out so this
is not always stored, resulting in a situation where the login count
can be significantly higher than the logout count; logout\_count is a
discrete quantitative variable. 

\textbf{reset\_profile\_count} the number of times a learner reset their
profile. Resetting the profile causes the status of all viewed and
learned concepts to be reset and the trafficlight menu will therefore
also be reset to its initial state. Reset\_profile\_count is a discrete
quantitative variable. 

\textbf{view\_concept\_count}, like totalconceptcount above, is another
measure of course concepts viewed. view\_concept\_count is a more
accurate figure than totalconceptcount because totalconceptcount counts
every time that the view concept page is loaded, but sometimes this
simply results in the menu being displayed. view\_concept\_count
includes only the times that learning material was actually presented
to the learner. This figure should be used in preference to
totalconceptcount for analysis, though an interesting exercise might be
to compare the statistics for the two variables. Like
totalconceptcount, view\_concept\_count is a discrete quantitative
variable. 

\textbf{view\_test\_count} is the number of times the learner clicked on
the {\textquotedbl}Test Me{\textquotedbl} link in the menu to view the
available tests. It is a discrete quantitative variable. 

\subsubsection[Survey Responses ]{\textrm{Survey Responses} }
The survey questions were answer on a scale from Strongly Agree to
Stronly Disagree. There is also a {\textquotedbl}No Answer
Given{\textquotedbl} category where learners left a response blank. 

~ 

 [Warning: Image not found]  

\bigskip

While all learners have data available for all of the system log
variables described above, the survey response data has some gaps. Not
all learners completed an end{}-of{}-course survey (many of these were
not asked to complete the survey so we can infer nothing from their
failure to complete it). This means that some of the learners have no
data available for the variables described below. Missing data such as
this is marked by a question mark symbol in our data file. 

It is important to note that a response category of ``No Answer Given''
is not the same as missing data (denoted by a ``?''). The former
indicates that the learner completed the post{}-course survey, but
chose to leave some questions blank. The latter means that the learner
did not complete the post{}-course survey (often because he or she was
a remote learner and we did not have an opportunity to forward the
survey to him/her). 

All survey responses are categorical or qualitative data and because
they are ordered from Strongly Disagree to Strongly Agree they can be
considered as ordinal data. The ``No Answer Given'' category does not
neatly fit into an ordinal view of the data, but there are not many
responses in this category. Where they do exist it might be plausible
to interpret them as actual indicating disagreement with the statement
presented. However, the question of how to view ``No Answer Given''
responses is better dealt with on a per{}-question basis. 


\bigskip

\textbf{elgrid{}-easy} gives the responses to the statement
{\textquotedbl}The eLGrid system is easy to use{\textquotedbl} 

\textbf{future{}-elgrid} gives the responses to the statement
{\textquotedbl}If I need to learn about Grid technologies in future I
would use the eLGrid eLearning system{\textquotedbl} 

\textbf{future{}-use{}-tools} gives the responses to the statement
{\textquotedbl}In the future I will probably use the technologies
taught in this course{\textquotedbl} 

\textbf{prac{}-apply{}-concepts} gives the responses to the statement
{\textquotedbl}The practical exercises helped me to understand and
apply the course concepts{\textquotedbl} 

\textbf{prac{}-env{}-easy} gives the responses to the statement
{\textquotedbl}The practical environment is easy to use{\textquotedbl} 

\textbf{prac{}-good{}-eval} gives the responses to the statement
{\textquotedbl}The practical exercises were a good way for me to
evaluate my knowledge{\textquotedbl} 

\textbf{practical{}-access} gives the responses to the statement
{\textquotedbl}The practical environment was easy to
access{\textquotedbl}\newline
\textbf{practical{}-understood} gives the responses to the statement
{\textquotedbl}The practical instructions were easy to
understand{\textquotedbl} 

\textbf{tests{}-good{}-eval} gives the responses to the statement
{\textquotedbl}The tests were a good way for the system to evaluate my
knowledge{\textquotedbl} 

\textbf{theory{}-comp{}-prac} gives the responses to the statement
{\textquotedbl}The theory in this course supported and complemented the
practical exercises{\textquotedbl} 

\textbf{too{}-easy} gives the responses to the statement
{\textquotedbl}The course was too simple, explaining things that I
already knew{\textquotedbl} 

\textbf{too{}-hard} gives the responses to the statement
{\textquotedbl}The course expected too much prior
knowledge{\textquotedbl} 

\textbf{traffic{}-light{}-prac} gives the responses to the statement
{\textquotedbl}It would be useful if the traffic{}-light indicators
took into account the results of my practical exercises as well as my
test answers{\textquotedbl} 

\textbf{traffic{}-understood} gives the responses to the statement
{\textquotedbl}I understood how my test answers affected the
traffic{}-light indicators{\textquotedbl} 

\textbf{trafficlights{}-nav} gives the responses to the statement
{\textquotedbl}The traffic{}-light indicators helped me to navigate the
course{\textquotedbl} 

\textbf{tutor{}-req} gives the responses to the statement
{\textquotedbl}The presence of a tutor was necessary for me to complete
this course, I would have had trouble if I was trying to use this
course on my own{\textquotedbl} 

\subsubsection[Other Data ]{\textrm{Other Data} }
The following variables are available for some learners. 

\textbf{projectmark} is the final mark given for a practical project
using technologies taught via the eLGrid system. This is only available
for a subset of the learners who used the eLGrid system as part of
their undergraduate studies. Marks are given as a percentage and are
treated as a continuous quantitative variable. 

\textbf{elgridmark} is a mark based on the correct completion of the
test questions in the system. Five concepts had associated tests and
20\% was given for each concept, some learners were docked marks
because the logs showed that they had gone back into the
{\textquotedbl}Test Me{\textquotedbl} screen repeatedly and used trial
and error to get the correct answer (i.e. they had many incorrect
answers before they finally got the correct answer). Marks are given as
a percentage and are treated as a continuous quantitative variable. 

\textbf{totalmark} is a combination of the weighted project and elgrid
marks. This is also a percentage and is treated as a continuous
quantitative variable. 

\textbf{gender} gives the the learner{\textquotesingle}s gender, it is
not known for all learners because many are remote learners about whom
we have little information. Gender is a nominal variable. 

\textbf{level} gives the learner{\textquotesingle}s level (undergrad,
postgrad, postdoc, academic staff (e.g. lecturer, professor, etc.)). As
with gender, this variable is not known for all learners. Level is a
nominal variable. 

\subsection{Questions and Analyses }
The purpose of our analysis is to determine whether there are patterns
to how learners use the eLGrid system. For example, do learners tend to
spend a similar amount of time viewing course concepts, completing
tests, etc., or is there a lot of variation in how they use these
features of the system. 

For those learners who have completed the end{}-of{}-course survey we
will also look at the relationships between their satisfaction scores
and their usage patterns. For example, is there a relationship between
high satisfaction and time spent viewing concepts? Is there a
relationship between the number of times the tests were viewed or
correctly answered and the learners{\textquotesingle} answers to the
question {\textquotedbl}The tests were a good way for the system to
evaluate my knowledge{\textquotedbl}, or {\textquotedbl}I understood
how my test answers affected the traffic{}-light
indicators{\textquotedbl}. 

We will attempt to look at the categorical variables Gender and Level
and see if there are differences in any of the other variables when
broken out by category. 

\subsection[Performing the Analysis ]{\textrm{Performing the Analysis} }
The sample size is 75 so for all variables \textit{n} will be 75, except
for those situations where outliers are being deliberately excluded
from the analysis.

\textbf{Survey Responses}\newline
The survey responses are ordinal variables; however, they have two
possible values {\textquotedbl}Survey Unanswered{\textquotedbl} and
{\textquotedbl}No Answer Given{\textquotedbl} which are not actually on
an ordinal scale but are simply nominal categories.~ As such, it is
necessary to perform some analysis on these variables ignoring these
two answers and looking only at the ordinal values.~ For example, while
we can look at the mode including all responses, only a minority of
learners completed the surveys, and thus the mode will always be
{\textquotedbl}Survey Unanswered{\textquotedbl}.~ This is not
particularly useful and so it is best to exclude these missing
datapoints when getting the mode.~ Similarly, the median can be
calculated, but it is only really meaningful for the ordinal values
(which, recall, correspond to a scale from 1 {}- Strongly Disagree to 5
{}- Strongly Agree permitting some limited numeric analysis).~ We will,
thus, exclude both {\textquotedbl}Survey Unanswered{\textquotedbl} and
{\textquotedbl}No Answer Given{\textquotedbl} when getting the mode.~
The mean may not be appropriate for data on an ordinal scale and it is
certainly inappropriate for purely categorical data, so we may simply
omit this measurement.\newline
While we have excluded the categorical values in some of the
calculations when we go to graph these variables it may make sense to
include all the data.\newline
It might also be useful to look simply at whether responses are above or
below 3 {\textquotedbl}Neither Agree nor Disagree{\textquotedbl} as
this will tell us more generally whether learners agree or disagree
with the statements.

\subsection{Structure of this report}
We will look at related data together and thus, the next five chapters
deal with particular themes of interest in the eLearning system. 

Chapter 2 looks at how learners access the system and the learning
materials.~ It includes analysis of variables such as login\_count,
total\_concept\_count, totalconcepttime, etc.

Chapter 3 looks at how learners use the test functionality while chapter
4 examines the practical exercises and learners{\textquotesingle} use
and opinions of them.\newline
Learners overall views of the eLGrid eLearning sytem are examined in
chapter 5, while chapter 6 looks at those variables which contain the
learners final marks, where these are available to us.

Chapter 7 looks at the issue of data normality and the applicability of
the normal model to our data. 

Chapter 8 graphs some variables against each other in order to look for
relationships and correlations in the data. 

Finally chapter 9 presents our conclusions. 

The data file and a time series analysis graph are included as
appendices. 

\section[Chapter 2: Accessing the eLearning application and the learning
materials ]{Chapter 2: Accessing the eLearning application and the
learning materials }
\newline
Several of the variables measured have to do with the frequency of
access to the system or specifically to the learning materials, or the
time spent at these tasks.~ For example, totalconcepttime measures the
time spent viewing course materials which login\_count measures the
number of time learners logged into the system.~ These variables all
tell use something about the way that learners access the system.

{\bfseries
Total Concept Time}

The total concept time variable has the following summary statistics:

~ 



\begin{longtable}[l]{|p{2.8490598in}|p{2.8490598in}|}
\hline
x[305?] 
&
71491.76 
\\\hline
s 
&
343086.4137 
\\\hline
s{\texttwosuperior} 
&
117708287265.528 
\\\hline
median 
&
3738 
\\\hline
min 
&
0 
\\\hline
max 
&
2528613 
\\\hline
range 
&
2528613 
\\\hline
IQR 
&
10417 
\\\hline
\end{longtable}
From this we can see that there is a huge variance and range in this
data.~ This may make the data difficult to interpret, but looking at
the relatively low median of 3738 suggests that most values may be
clustered to the left.~ We will see this more clearly when we graph
this variable.\newline
We next attempted to find outliers in the data.~ Using the notion that
any point below Q1 {}- 1.5(IQR) or above Q3 + 1.5(IQR) was a suspect
outlier, we found no outliers on the left but eight on the right.:
38164, 42683, 48723, 72530, 92493, 732600, 1462891 and 2528613.~ The
final point 2528613 is from the row with learnerid = 2,~ which was
previously identified as test data and could have been safely excluded
from the analysis.~ However, these other outliers are valid data and
they simply indicate that some learners spend a lot more time viewing
course materials than others.\newline
We graphed this variable as a stemplot, a histogram and a boxplot.~ The
first graph below is the stemplot of the raw data in seconds.~ Our
stems in this graph are in units of 1000 seconds as there were a large
number of~ datapoints in the thousands.~ This gives us a good view of
the distribution of the data; however, as you can see the data is
strongly skewed to the right; so much so in fact that it was not
possible to include all points in the plot and thus the plot jumps at
certain points. 


\bigskip

~0 {\textbar} 0~ 0~ 0~ 0~ 0~ 0~ 0~ 0~ 0~ 0~ 14~ 18~ 57~ 155~ 184~ 214~
214~ 275~ 350~ 759\newline
~1 {\textbar} 246~ 288~ 419~ 517~ 746\newline
~2 {\textbar} 256~ 625~ 629~ 849\newline
~3 {\textbar} 013~ 191~ 254~ 307~ 397~ 553~ 636~ 669~ 738~ 774\newline
~4 {\textbar} 806\newline
~5 {\textbar} 104~ 246\newline
~6 {\textbar} 516~ 904~ 906~ 915\newline
~7 {\textbar} 368~ 808\newline
~8 {\textbar} 257~ 735~ 774~ 895\newline
~9 {\textbar}\newline
10 {\textbar} 170~ 450~ 534~ 540~ 767~ 871\newline
11 {\textbar} 094~ 679~ 715\newline
12 {\textbar} 143~ 548~ 761\newline
13 {\textbar}\newline
14 {\textbar}\newline
15 {\textbar}\newline
16 {\textbar}\newline
17 {\textbar}\newline
18 {\textbar} 597\newline
19 {\textbar} 072\newline
20 {\textbar}\newline
21 {\textbar}\newline
22 {\textbar}\newline
23 {\textbar} 663\newline
24 {\textbar}\newline
25 {\textbar} ~\newline
26 {\textbar}\newline
27 {\textbar}\newline
28 {\textbar}\newline
29 {\textbar}\newline
30 {\textbar}~ ~\newline
31 {\textbar}\newline
32 {\textbar}\newline
33 {\textbar}\newline
34 {\textbar}\newline
35 {\textbar}\newline
36 {\textbar}\newline
37 {\textbar}\newline
38 {\textbar} 164\newline
39 {\textbar}\newline
40 {\textbar}\newline
41 {\textbar}\newline
42 {\textbar} 683\newline
43 {\textbar}\newline
44 {\textbar}\newline
45 {\textbar}\newline
46 {\textbar}\newline
47 {\textbar}\newline
48 {\textbar} 723

.\newline
.\newline
.\newline
70 {\textbar}\newline
71 {\textbar}\newline
72 {\textbar} 530\newline
.\newline
.\newline
.\newline
92 {\textbar} 493\newline
.\newline
.\newline
.\newline
732 {\textbar} 600\newline
.\newline
.\newline
.\newline
1462 {\textbar} 891\newline
.\newline
.\newline
.\newline
2528 {\textbar} 613


\bigskip

As mentioned the data for this variable is in seconds.~ Converting to
hours would give us smaller and more easily understandable units to
work with.~ However this could cause us to lose precision.~ In this
case the difficulty of representing the data together on a single plot
because of its large range and outliers makes it necessary to group the
data anyway in order to properly view it; so converting to hours is
acceptable.~ The following is a histogram showing the data when
converted to hours. 

~ 

 [Warning: Image not found]  
As you can see, because of outliers almost all of the data is grouped
together in the 0{}-100 hours bar.~ This is not particularly useful, so
we created a new version of this histogram which excludes the three
outliers above 200 hours. 


\bigskip

 [Warning: Image not found]  

\bigskip

This is a slightly more useful graph showing that most learners spend
between 0 and 5 hours viewing concept materials on the system, while a
small number spend between 5 and 15 hours and a smaller number again
spend between 20 and 30 hours. 

In order to attempt to show the outliers and the clustering of the
majority of learners in a single graph we have created a boxplot;
however, as you can see, because of the huge difference in values it
was necessary to make the scale discontinuous in order to fit the
outliers into the plot.~ Thus the rightmost part of the graph is not
drawn to scale, it merely shows us that very large outliers exist.~
Once again the data here is represented in seconds, but the graph would
look the same in any units.


\bigskip

~ 

 [Warning: Image not found]  

\bigskip

\textbf{Total Concept Count} 

When row two is included we get a mean of~ 117.5 as opposed to 83.83
when w exclude row two.~ We decided to exclude row two from the
analysis this time as it was very far away from the center.

Excluding row two we got the following summary statistics 

~ 



\begin{longtable}[l]{|p{2.8490598in}|p{2.8490598in}|}
\hline
x[305?] 
&
84.96 
\\\hline
s 
&
97.16
\\\hline
s{\texttwosuperior} 
&
9440.23
\\\hline
median 
&
50.5 
\\\hline
min 
&
0
\\\hline
max 
&
565 
\\\hline
range 
&
565 
\\\hline
IQR 
&
126 
\\\hline
\end{longtable}
The number of times that learners viewed the material has a much smaller
range and variation than the time that they spent viewing them. 

For this variable we found two outliers (again on the right) 377 and 565
(recall that we have excluded the learnerid = 2 value of 2518 from the
analysis) 


\bigskip

The following stemplot gives us a good view of the distribution of
values for the totalconceptcount variable.~ As you can see the curve is
not normal and there is considerable skew to the right (though,
considerably less skew is evident in the counts than the times). 


\bigskip

~0 {\textbar} 0 0 0 0 0 0 0 0 0 0 1 3 3 3 5 9 9 9\newline
~1 {\textbar} 1 3 5\newline
~2 {\textbar} 4 4 8\newline
~3 {\textbar} 4 4 5 6 8 9\newline
~4 {\textbar} 2 5 5 6 7 8\newline
~5 {\textbar} 0 1 5 8\newline
~6 {\textbar} 2 4 7 8\newline
~7 {\textbar} 6\newline
~8 {\textbar} 3 9\newline
10 {\textbar} 2 3 5 8 9\newline
11 {\textbar} 6\newline
12 {\textbar} 0 3\newline
13 {\textbar} 7 7 9\newline
14 {\textbar} 0 9\newline
15 {\textbar} 6\newline
16 {\textbar} 7\newline
17 {\textbar} 1 6\newline
18 {\textbar} 7 8\newline
19 {\textbar} 4 5 8\newline
20 {\textbar} 1\newline
21 {\textbar}\newline
22 {\textbar}\newline
23 {\textbar}\newline
24 {\textbar}\newline
25 {\textbar}\newline
26 {\textbar} 1\newline
27 {\textbar}\newline
28 {\textbar}\newline
29 {\textbar} 4\newline
30 {\textbar}\newline
31 {\textbar}\newline
32 {\textbar}\newline
33 {\textbar}\newline
34 {\textbar}\newline
35 {\textbar}\newline
36 {\textbar}\newline
37 {\textbar} 7\newline
38 {\textbar}\newline
39 {\textbar}\newline
40 {\textbar}\newline
41 {\textbar}\newline
42 {\textbar}\newline
43 {\textbar}\newline
44 {\textbar}\newline
45 {\textbar}\newline
46 {\textbar}\newline
47 {\textbar}\newline
48 {\textbar}\newline
49 {\textbar}\newline
50 {\textbar}\newline
51 {\textbar}\newline
52 {\textbar}\newline
53 {\textbar}\newline
54 {\textbar}\newline
55 {\textbar}\newline
56 {\textbar} 5


\bigskip

~ 

 [Warning: Image not found]  



\bigskip

~ 

 [Warning: Image not found]  

\bigskip

~ 

~ 

\textbf{view\_concept\_count} 


\bigskip

\begin{longtable}[l]{|p{1.6712599in}|p{1.7309599in}|}
\hline
mean 
&
137.17 
\\\hline
s 
&
349.17 
\\\hline
s\textsuperscript{2} 
&
121920.74 
\\\hline
median 
&
63.00 
\\\hline
min 
&
0.00 
\\\hline
max 
&
2972.00 
\\\hline
range 
&
2972.00 
\\\hline
IQR 
&
145.00 
\\\hline
\end{longtable}
~ 

~ 

~ 

~ 

~ 

~ 

~ 

~ 

~ 

~

{\bfseries
Login Count}

This variable is a measure of how many times the learners logged in in
order to view the available tests.~ The learnerid = 2 value of 199 has
been excluded from the analysis. 

~ 



\begin{longtable}[l]{|p{3.5809598in}|p{3.5809598in}|}
\hline
x[305?] 
&
5.27
\\\hline
s
&
5.261119 
\\\hline
s{\texttwosuperior}
&
27.67938 
\\\hline
median 
&
3.00
\\\hline
min
&
1.00
\\\hline
max
&
28.00
\\\hline
range
&
27.00
\\\hline
IQR
&
7.75
\\\hline
\end{longtable}
 [Warning: Image not found]  
 [Warning: Image not found]  

\bigskip

{\bfseries
Logout Count}

This variable is a measure of how many times the learners logged in in
order to view the available tests.~ The learnerid = 2 value of 199 has
been excluded from the analysis. 

~ 



\begin{longtable}[l]{|p{3.5809598in}|p{3.5809598in}|}
\hline
x[305?] 
&
1.81
\\\hline
s
&
2.27
\\\hline
s{\texttwosuperior}
&
5.17
\\\hline
median 
&
1.00
\\\hline
min
&
0.00
\\\hline
max
&
12.00
\\\hline
range
&
12.00
\\\hline
IQR
&
1.75
\\\hline
\end{longtable}

\bigskip

 [Warning: Image not found]  
 [Warning: Image not found]  
\section[Chapter 3: The {\textquotedbl}Test Me!{\textquotedbl} feature
]{Chapter 3: The {\textquotedbl}Test Me!{\textquotedbl} feature }
\newline
The {\textquotedbl}Test Me!{\textquotedbl} link in the eLGrid course
menu allows learners to take tests to check their progress.~ The tests
not only give an indication of whether or not the learner has
understood the material, they also feed into the adaptive
{\textquotedbl}trafficlight{\textquotedbl} menu system.~ When a learner
views a concept the trafficlight icon for that concept will change from
green to amber.~ Once the learner completes the associated tests the
icon will change to lilac.\newline
The eLGrid system logs data about learners{\textquotesingle} use of the
{\textquotedbl}Test Me!{\textquotedbl} feature.~ For example, how many
times the learner views the tests is logged along with the number of
correct and incorrect test answers which they give.\newline
The end of course survey also asks some questions specifically about the
{\textquotedbl}Test Me!{\textquotedbl} feature.~ Together these
variables tell us about the pattern of usage of the tests and also the
learners{\textquotesingle} opinions of and attitudes to the tests and
their usefulness.

{\bfseries
View Test Count}

This variable is a measure of how many times learners clicked on the
{\textquotedbl}Test Me{\textquotedbl} button in order to view the
available tests.~ The learnerid = 2 value of 122 has been excluded from
the analysis. 

~ 



\begin{longtable}[l]{|p{3.5809598in}|p{3.5809598in}|}
\hline
x[305?] 
&
4.09 
\\\hline
s
&
9.64 
\\\hline
s{\texttwosuperior}
&
92.96
\\\hline
median 
&
1
\\\hline
min
&
0
\\\hline
max
&
72 
\\\hline
range
&
72 
\\\hline
IQR
&
6 
\\\hline
\end{longtable}
These values give us as outliers any datapoint below {}-9 or above15.~
There are no outliers to the left and four to the right 19, 22, 29, 72.


As the majority of datapoints for this variable are below 10 a stemplot
is not particularly useful 

~0 {\textbar} 0 0 0 0 0 0 0 0 0 0 0 0 0 0 0 0 0 0 0 0 0 0 0 0 0 0 0 0 0
0 0 0 1 1 1 1 1 1 1 1 1 1 1 1 1 1 2 2 2 2 3 3 3 3 3 6 6 6 6 7 7 8 8 9 9
9\newline
~1 {\textbar} 0 0 0 3 9\newline
~2 {\textbar} 2 9\newline
~3 {\textbar}\newline
~4 {\textbar}\newline
~5 {\textbar}\newline
~6 {\textbar}\newline
~7 {\textbar} 2\newline
~8 {\textbar}\newline
~9 {\textbar}\newline
10 {\textbar}\newline
11 {\textbar}\newline
12 {\textbar}2


\bigskip

It does show us very quickly that the data is not normal and is skewed
to the right.~ We can see this again in the histogram (this time we
have excluded the datapoint 122 which is from the test user with
userid=2) and we have also excluded the outlier value 72 for ease of
graphing.~ Both of these values were included in the stemplot.

~ 

 [Warning: Image not found]  

\newline

For the boxplot we have included the outlier 72 but excluded the test
value 122.~ Note that while 72 is an obvious outlier which can be
identified intuitively by looking at the data, using the formula Q3 +
1.5(IQR) identifies four potential outliers.~ However, the first three
of these might reasonably be considered not to be outliers at all.

 [Warning: Image not found]  

\bigskip

\textbf{correct\_answer\_count} 


\bigskip

\begin{longtable}[l]{|p{1.6712599in}|p{1.7309599in}|}
\hline
mean 
&
12.75 
\\\hline
s 
&
35.83 
\\\hline
s\textsuperscript{2} 
&
1283.79 
\\\hline
median 
&
0.00 
\\\hline
min 
&
0.00 
\\\hline
max 
&
188.00 
\\\hline
range 
&
188.00 
\\\hline
IQR 
&
9.50 
\\\hline
\end{longtable}

\bigskip


\bigskip


\bigskip

\textbf{reset\_profile\_count} 


\bigskip

\begin{longtable}[l]{|p{1.6712599in}|p{1.7309599in}|}
\hline
mean 
&
0.32 
\\\hline
s 
&
1.72 
\\\hline
s\textsuperscript{2} 
&
2.95 
\\\hline
median 
&
0.00 
\\\hline
min 
&
0.00 
\\\hline
max 
&
14.00 
\\\hline
range 
&
14.00 
\\\hline
IQR 
&
0.00 
\\\hline
\end{longtable}

\bigskip

\newline


\section[Chapter 4: Practical Exercises ]{Chapter 4: Practical Exercises
}
\newline
The eLGrid system teaches users about Grid technologies.~ As such, many
courses include a practical component where learners have an
opportunity to try out their knowledge using the tools and environments
about which they have been learning.~ This practical component is an
important part of the eLGrid system which aims to integrate, as far as
is possible, with the Grid envinronment in order to make is easy to
complete the practical exercises.\newline
The learners attitudes about the practical exercises and environments
are captured in the end of course survey.\newline
\textbf{prac{}-good{}-eval\newline
}The prac{}-good{}-eval variable gives learner responses to the
statement {\textquotedbl}The Practical exercises were a good way for me
to evaluate my knowledge{\textquotedbl}.~ The majority of learners did
not complete the survey ({\textquotedbl}Survey
Unanswered{\textquotedbl}).~ Of these most were not actually asked to
complete the survey, or the survey that they completed did not contain
this question, so their lack of response tells us nothing about their
opinions.\newline
The {\textquotedbl}No Answer Given{\textquotedbl} response on the other
hand indicates that the learner completed the survey but intentionally
(we assume) left this answer blank.~ We must decide how to handle these
blank answers and this is best done on a question{}-by{}-question
basis.~ For example, in this case, the learner may not have completed
any practical exercises or perhaps they completed a course in the
system which contained few or no practical exercises.~ In such a case
the learner might leave this answer blank.~ It does not seem likely
that a blank answer could be interpreted as either agreement or
disagreement with the statement, although it is possible that we could
consider it to mean that they have no opinion either way, i.e. they
{\textquotedbl}Neither Agree nor Disagree{\textquotedbl} with the
statement.~ However, in this case it might be more illustrative to
treat {\textquotedbl}No Answer Given{\textquotedbl} as a completely
separate category.\newline
For the summary statistics we have left {\textquotedbl}Survey
Unanswered{\textquotedbl} and {\textquotedbl}No Answer
Given{\textquotedbl} out when calculating the summary statistics, but
we have included them in the bar graph and exluded just the
{\textquotedbl}Survey Unanswered{\textquotedbl} values from the pie
chart (as they would otherwise dominate the chart).



\begin{longtable}[l]{|p{3.5809598in}|p{3.5809598in}|}
\hline
median (excluding {\textquotedbl}Survey Unanswered{\textquotedbl} and
{\textquotedbl}No Answer Given{\textquotedbl})
&
4
\\\hline
Mode
&
4
\\\hline
min
&
3
\\\hline
max
&
5
\\\hline
range
&
2
\\\hline
IQR
&
1
\\\hline
\end{longtable}
From the above we can see that there is general agreement with the
statement {\textquotedbl}The practical exercises were a good way for me
to evaluate my knowledge{\textquotedbl}.~ The median is 4 (Somewhat
Agree), as is the mode and no answers were less than 3, i.e. no
learners disagreed with the statement.~ The range of responses is also
quite low suggesting that most learners had a similar opinion of the
efficacy of the practical exercises.\newline
\newline
 [Warning: Image not found]  

 [Warning: Image not found]  

\bigskip

\section[Chapter 5: Overall views of the eLGrid system ]{Chapter 5:
Overall views of the eLGrid system }
\newline
Some of the data gives us general information about how learners felt
about the eLGrid eLearning system; was it easy to use? was it easy to
access? etc.~ This sort of information can be very useful to the system
developers, showing them areas where the system could use
improvement.\newline
In this chapter we will also look at the data relating to the adaptive
traffic{}-lights menu system.\newline
\newline
\textbf{traffic{}-lights{}-nav} 



\begin{longtable}[l]{|p{3.5809598in}|p{3.5809598in}|}
\hline
median (excluding {\textquotedbl}Survey Unanswered{\textquotedbl} and
{\textquotedbl}No Answer Given{\textquotedbl}) 
&
3.5
\\\hline
Mode 
&
4
\\\hline
min
&
1
\\\hline
max
&
5
\\\hline
range
&
4
\\\hline
IQR
&
2
\\\hline
\end{longtable}
Responses to the statement {\textquotedbl}The traffic{}-light icons
helped me to navigate the course{\textquotedbl} was quite varied, with
an overall range of 4 and an IQR of 2.~ The median of 3.5 suggests that
learners are slightly more inclined to agree with this statement than
to disagree, as does the mode of 4.\newline
\newline
\textbf{tutor{}-req} 



\begin{longtable}[l]{|p{3.5809598in}|p{3.5809598in}|}
\hline
{\mdseries
median (excluding {\textquotedbl}Survey Unanswered{\textquotedbl} and
{\textquotedbl}No Answer Given{\textquotedbl}) }
&
{\mdseries
4}
\\\hline
{\mdseries
mode}
&
{\mdseries
4}
\\\hline
{\mdseries
min}
&
{\mdseries
1}
\\\hline
{\mdseries
max}
&
{\mdseries
5}
\\\hline
{\mdseries
range}
&
{\mdseries
4}
\\\hline
{\mdseries
IQR}
&
{\mdseries
2}
\\\hline
\end{longtable}
Most learners agreed with the statement {\textquotedbl}The presence of a
tutor was necessary for me to complete this course{\textquotedbl} as
suggested by the median and mode of 4. However, here again we see quite
a spread of answers with a minimum of 1 (Strongly Disagree), a maximum
of 5 (Strongly Agree) and a range of 4 and IQR of 2.~ This response is
not what we had hoped to see as the eLearning system should ideally be
capable of being used by remote learners without any tutor.~ There were
some technical problems during one of the workshops where survey
responses were gathered which might have contributed to these results,
or the system{\textquotesingle}s content or user interface may need
some development in order to make it more suitable for remote
learners.\newline
\textbf{\newline
trafficlights{}-nav} 

 [Warning: Image not found]  

 [Warning: Image not found]  
\textbf{tutor{}-req} 

 [Warning: Image not found]  
 [Warning: Image not found]  

\bigskip

\section[Chapter 6: Final Marks ]{Chapter 6: Final Marks }
\newline
As mentioned in the introduction, a small number of the learners were
undergraduate students for whom we have information about their final
scores.~ This data is available only for 14 learners out of the 75.

\textbf{elgridmark} 


\bigskip

\begin{longtable}[l]{|p{1.6712599in}|p{1.7309599in}|}
\hline
mean 
&
95 
\\\hline
s 
&
10.91 
\\\hline
s\textsuperscript{2} 
&
119.23 
\\\hline
median 
&
100 
\\\hline
min 
&
60 
\\\hline
max 
&
100 
\\\hline
range 
&
40 
\\\hline
IQR 
&
10 
\\\hline
\end{longtable}

\bigskip

\textbf{projectmark} 

~ 


\bigskip

\begin{longtable}[l]{|p{1.6712599in}|p{1.7309599in}|}
\hline
mean 
&
75.00
\\\hline
s 
&
10.91 
\\\hline
s\textsuperscript{2} 
&
119.23 
\\\hline
median 
&
75.00
\\\hline
min 
&
60.00
\\\hline
max 
&
90.00 
\\\hline
range 
&
30.00 
\\\hline
IQR 
&
10.00 
\\\hline
\end{longtable}
\section[Chapter 7: Normality of the Data ]{Chapter 7: Normality of the
Data }
\newline
Much of the data collected does not appear to be particularly normal in
distribution.~ We are measuring aspects of human behaviour, which do
often fall into a normal distribution.~ However, in this case many of
our measurements would not be expected to have a normal distribution.~
For example, the time spent viewing course material is an open{}-ended
scale beginning at 0 but continuing unbounded.~ Our learner accounts
were created at different times and learners have thus had different
periods of time available to them to use the sytem.~ Furthermore, there
are seven different courses available in the system and learners have
studied different ones, some completing just one course, and others
completing many.~ Yet other learners seem to have requested accounts
more out of curiousity than any real wish to use the system.~ These
learners have only logged in a few times and may not actually have
completed any courses.\newline
We wanted to know if we would see a more normal distribution when we
viewed a subset of the learners, all of whom had completed the same
course (and no other courses), over the same period.~ One such subset
is data collected from a set of learners who used the system as part of
their undergraduate course. Their accounts were all issued at the same
time and they began using the system at the same time.~ They used the
system for a number of weeks until their project deadline and exams
were over, after which they did not return to the system.~ All learners
completed just one course (although a small number looked briefly at
some of the other courses too).\newline
Even with this subset we still see a huge variety in the data.~ For
example, the standard deviation of totalconcepttime for all of the data
was 343086.41, but as you can see from the table below when we isolate
just this subset we find that they actually have a larger standard
deviation and variance.\newline
\newline
\textbf{totalconcepttime for a subset of the data all using the same
course}



\begin{longtable}[l]{|p{3.5809598in}|p{3.5809598in}|}
\hline
x[305?] 
&
123054.1 
\\\hline
s 
&
386306.9 
\\\hline
median 
&
12654.5 
\\\hline
min
&
3307 
\\\hline
max
&
1462891 
\\\hline
range
&
1459584 
\\\hline
IQR
&
14297.5 
\\\hline
\end{longtable}
\newline
As you can see from the high values of s and the IQR, and the huge
difference between the mean and median, this data is likely to have
some outliers skewing the data towards the right.~ A graph confirms
this, as you can see one outlier to the far right skews the data.

 [Warning: Image not found]  
If we remove this outlier we get the following histogram which is still
not a normal distribution but rather it is denser around the lower
values with a lower frequency as the value increases. 

 [Warning: Image not found]  
\newline
\newline
\textbf{totalconceptcount for a subset of the data all using the same
course}



\begin{longtable}[l]{|p{3.5809598in}|p{3.5809598in}|}
\hline
x[305?] 
&
197.5714 
\\\hline
s 
&
121.5804 
\\\hline
median 
&
177.5 
\\\hline
min
&
83 
\\\hline
max
&
565 
\\\hline
range
&
482 
\\\hline
IQR
&
73.75 
\\\hline
\end{longtable}
\newline
Concept count appears to have a distribution closer to normal. When we
look only at our subset of users, however, it is still not a
particularly normal distribution.

 [Warning: Image not found]  

\bigskip

{\bfseries
log{}-normal data}

We wondered if it might be possible to find a normal distribution of the
logs of the data and we created histograms of the logs of several of
our variables.~ The complete dataset included a number of 0 values,
which had to be removed in order to get the logs.~ This put some
constraints on which variables we could use in such an analysis.~ For
some variables, 0 values could safely be ignored, for example, a 0
value for the totalconcepttime or totalconceptcount variables means
that the learner has never viewed any learning materials in the
system.~ These learners are most likely those who had accounts created
but never used them and it is quite valid to ignore them in any
analysis.~ However, with some of the other variables it might not be
acceptable to discard 0 values as they may be telling us something
about the learner{\textquotesingle}s use of the data.

The graph below shows the frequency distribution of the log of
totalconcepttime for all learners, excluding those who had spend no
time and therefore had a 0 value.~ While the raw data for
totalconcepttime was clearly not normal, when we looked at the logs of
the data we found a distribution much closer to normal.

 [Warning: Image not found]  

\bigskip

We then performed a Shapiro{}-Wilk test using the R statistical package
and found a W statistic of 0.9664 and a p{}-value of 0.07465.~ The
relatively large W statistic and p{}-value greater than our alpha value
(here 95\% or 0.05), tells us that we cannot reject the null hypothesis
that the data comes from a normal population.~ This does not
necessarily tell us that the log data is normal, but it does not rule
it out and so we may be able to apply other analyses to the log data
which are only appropriate for normally distributed data.


\bigskip

\section[Chapter 8: Relationships and Correlations ]{Chapter 8:
Relationships and Correlations }
Correlation is a comparable method used to assess the association
between continuous variables. The standard method leads to a quantity
\textit{r }which can take any value from {}-1 to +1. This correlation
coefficient \textit{r }measures the degree of `straight{}-line'
association between the values of the two continuous variables. A value
of +1.0 or {}-1.0 is obtained if all the points in a scatter diagram
lie on a perfect straight line. The correlation between two variables
is positive if higher values of one variable are associated with higher
values of the other variable and negative if one variable tends to be
lower as the values of the other variable increase. A correlation of
around zero indicates that there is no linear relation between the
values of the two continuous variables. The correlation coefficient
\textit{r}~is a measure of the scatter of the points around the
underlying linear trend: the greater the spread of points the lower the
correlation. It can be calculated for any data set (Altman, 1991). 

Moore \& McCabe (2006) define correlation as the measurement of the
direction and strength of the linear relationship between two
quantitative variables. Suppose we have data on variables \textit{x}
and \textit{y }for \textit{n }individuals. The means and standard
deviations of the two variables are x bar and s\textsubscript{x} for
the x{}-values and y bar and s\textsubscript{y} for the y{}-values. The
correlation \textit{r }between \textit{x} and \textit{y} is: \textit{r}
= 1/\textit{n} {--} 1) ${\sum}$ (x\textsubscript{i} {--} x bar/
s\textsubscript{x} \textbf{) (}y\textsubscript{i }{}- y bar/
s\textsubscript{y} \textbf{)} 

~~ 

Altman warns that there is a restriction on the validity of the
associated hypothesis test ie the two variables must be observed on a
random sample of individuals and the data for at least one of the
variables must have a Normal distribution in the population~(Altman,
1991).\newline
\newline
\textbf{Concept time versus concept count\newline
}We expected a clear correlation between the number of concepts that a
learner viewed and the time spent viewing concepts, however as the
graph below shows this was not the case.~ Some learners viewed concepts
many times but only spent an average overall time viewing concepts,
while others spent a long time viewing concepts but did not view them
that frequently.~ This may come be due to natural variation in how long
the learners take to view the concept.~ Some might read the material
quickly but return several times while others may only view each
concept once but spend a long time reading and understanding the
material.

 [Warning: Image not found] \newline
\textbf{Login count versus concept count}

 [Warning: Image not found]  
\newline
There was a clearer relationship between the login count and the concept
count.~ Thus we can say that learners who returned to the system
regularly viewed more concepts on average than those who only logged in
a few times.~ This is what we would expect to see. Results also
indicate that when learners viewed concepts they tended to get the
correct answer (ie correlation of 0.4957708). One should also note that
there was a correlation between viewing the concept and the number of
times the profile was reset (ie correlation of 0.9472391). 

~ 

\newline
\textbf{View test count versus total concept count}\newline
We thought that we might see some relationship between the login count
and the number of times that the learners viewed and completed the
tests, however this was again not the case.~ A large number of learners
appeared to have viewed no tests even though they had logged in many
times. These learners may have completed courses with few or no tests
or they may simply not have been interested in using the
{\textquotedbl}Test Me!{\textquotedbl} function.~ More investigation
would be required to understand this.\newline
We did see some correlation, however, between the total concept count
and the number of times that the learners viewed the tests.~ While
there is a cluster of values with low view test counts there is also
some evidence of a linear relationship between the number of concepts
viewed and the number of tests viewed.~ This is another potential area
for further investigation. It might be possible to see a clearer
relationship if we take only the subset of learners who have actually
viewed some tests.

 [Warning: Image not found] \newline
\textbf{Totalconceptcount versus correct answer}\newline
The number of correct answers may also show some evidence of a
correlation with the total concept count.~ The graph below shows the
total concept count plotted against the number of correct test
answers.~ As you can see there is again a cluster around 0 correct
answers, but if we disregard these learners (who are probably largely
the same set as those above who did not view any tests) then there is a
trend for the two values to increase together, i.e. there is some
evidence of a linear relationship between these two variables, although
it appears less clear than that seen for the view test count.\newline


\bigskip

 [Warning: Image not found]  

\bigskip


\bigskip

{\bfseries
Correct versus Incorrect answers}

Interestingly, while one might expect an inverse relationship between
the number of correct and incorrect test answers, we actually saw the
opposite, as the graph below shows.~ Those learners who had a high
number of correct answers also tended to have a higher incorrect answer
count. A likely explanation for this is that the learners with higher
values for these variables simply took more tests than those with lower
values; thus they had more opportunity to get both correct and
incorrect answers.

 [Warning: Image not found] \newline
\newline
\textbf{Viewing different classes}\newline
We had two nominal variables in our data, so far we have done little
analysis of these.~ In this section we look briefly at these variables
and at how some of the other data varies when grouped by these
classifier variables.\newline
\newline
\newline
\textbf{Total Concept Count by Gender}

 [Warning: Image not found]  
\newline
\newline
\textbf{Total Concept Count by Level} 

 [Warning: Image not found]  
\newline
\textbf{View Test Count by Level} 

 [Warning: Image not found]  
\newline
\textbf{Correct Answer Count by Level} 

 [Warning: Image not found]  
\newline
\newline
\newline



\section{Chapter 9: Exploring differences between populations,
application of a t{}-test }
\newline
In chapter 7 we took a subset of our data, the group of learners who had
completed a particular course as part of their undergraduate studies,
and looked at whether their data followed a more normal distribution
than that of the other learners.~ In this chapter we will look again at
this subset of learners and compare them to the set of all other
learners to look for differences and similarities in how they used the
system.\newline
We have chosen to look at the logs of the totalconcepttime variable as
this variable may be normally distributed, because of this we can apply
a t{}-test and ask whether the mean of the logs
learners{\textquotesingle} times in the different groups are
statistically different, or if any difference that we see is merely due
to random variations.\newline
First we needed to split our sample into two distinct samples.~ The
first group was created by removing the 14 learners who completed the
course as part of their undergraduate degree, and all learners with a 0
value for the totalconcepttime variable.~ Removing the 0{}-values was
necessary as we will be working with logs of the data and zero values
produce infinite log values.~ This gave us a sample of 50 learners.~ We
also excluded the test data from the learner with user\_id = 2.\newline
The second group comprised the 14 undergraduate learners previously
mentioned.\newline
\newline
We assume that the logs of times for all users are normal (based on our
graph of the variable in chapter 7 and the Shapiro{}-Wilk test which we
performed).~ Looking at the graphs for our two subgroups we can see
that these also both appear fairly normal (the first sugroup looks more
normal than the second, but this makes sense as the sample size n is
larger for group 1). 

\newline
~ 

 [Warning: Image not found]  
 [Warning: Image not found]  
\newline
\newline
The statistics for the two groups are given in the table below:



\begin{longtable}[l]{|p{1.3851599in}|p{1.3851599in}|p{1.3851599in}|p{1.3851599in}|p{1.3851599in}|}
\hline
Sample
&
x{\textasciimacron}
&
s
&
s{\texttwosuperior}
&
n
\\\hline
Sample 1
&
7.992
&
1.987698
&
3.95
&
50
\\\hline
Sample 2
&
9.809
&
1.543944
&
2.38
&
14
\\\hline
\end{longtable}
\newline
We then formulated our hypotheses and calculated our t value as follows:

 [Warning: Image not found]  
Critical value for two{}-tailed test with 95\% confidence and 62 degrees
of freedom (table A2) is 2.0, thus the t{}-values for which we will
reject the null hypothesis that the two samples come from populations
with the same mean is t {\textless} {}-2 and t {\textgreater} 2 or the
area shaded orange in the graph below.

 [Warning: Image not found]  
\newline
Our t{}-value of 3.16 falls in the shaded region and thus we must reject
the null hypothesis that the two samples come from populations with the
same mean.~ What does this tell us about our two samples?~ It suggests
that there is a statistically significant difference in the logs of
time spent viewing concepts for those learners who studied one
particular course as part of their undergraduate studies and those who
used the eLGrid system in other ways, primarily as part of short
2{}-day to 1{}-week workshops.~ Further analysis might reveal other
interesting differences between these two groups.

\section[Chapter 9: Conclusions ]{\rmfamily Chapter 9: Conclusions }
\subsubsection[The eLGrid System]{\rmfamily The eLGrid System}
\subsubsection[Missing Data ]{\rmfamily Missing Data }
According to Altman (1991) missing is just an additional category for
categorical variables and so these individuals can be included in any
cross{}-tabulation. For continuous variables it is essential that
missing data are identified. It is worth thinking about why the data
are missing {--} in particular one ought to know if there is a reason
related to the nature of this study. 

~ 

Frequently values are missing essentially at random, for reasons not
related to a study. For example, some participants may not have been
asked a particular question.


\bigskip

\section[Chapter XXX: Ordinal Data and Likert Scales. ]{\rmfamily
Chapter XXX: Ordinal Data and Likert Scales. }
\newline
An ordinal scale of measurement represents an ordered series of
relationships or rank order. Examples are the results of a contest or
be measurements taken (often in questionnaire format) from participants
in an activity. Examples of such activities are participating in a
course, viewing a play or film or staying in a hotel, etc.\newline
\newline
Whereas this data is useful, it has some limitations, these are outlined
here.\newline
\newline
These scales do not represent a measurable quantity. In the results of a
contest, it is clear that first preformed better that all others, but
does not tell by how much or how far first was ahead of second, third,
etc. Further details of scores or other measurements are required for
this.\newline
\newline
Course participants may be asked to rate their experience in terms of
enjoyment, understanding of course material or benefit gained.\newline
\newline
Answers given are subjective. It is conceivable that a person who
thoroughly understood or enjoyed an activity may give a mid{}-scale
answer, while a person with a mediocre experience of the same activity
may rate that experience higher.\newline
\newline
Answering may also be subject to response biases. Examples of such
biases are acquiescence bias and social desirability bias. In
acquiesence bias, the respondent has a tendency to agree with all the
questions or to indicate a positive connotation. The designer of the
questions may decide to remove the possibility of giving a neutral
answer (e.g. Neither agree not disagree).\newline
\newline
Social desirability bias is a tendency to respond in a way which will be
perceived as being socially acceptable and or desirable. Surveys on
sensitive areas like religion, bigotry, domestic violence, etc. may
likely to be subject to this if questions are not properly
structured.\newline
\newline
The question used to obtain the ordinal answers in this data can be
termed a Likert scale. An typical Likert scale would be...\newline
\newline
~* Strongly disagree\newline
~* Disagree\newline
~* Neither agree nor disagree\newline
~* Agree\newline
~* Strongly Agree\newline
\newline
Often five ordered response levels are used although seven or nine
levels are also used.A recent empirical study[1] found that a 5{}- or
7{}- point scale may produce slightly higher mean scores relative to
the highest possible attainable score, compared to those produced from
a 10{}-point scale,\newline
\newline
The data examined in this work are responses to questionnaires which
used a 5 point Likert scale.

\section[Chapter YYY: Normality test on the projectmark variable (i.e. a
small sample). ]{\rmfamily Chapter YYY: Normality test on the
projectmark variable (i.e. a small sample). }
This data set consists of 14 data points. The (ranked) data set is as
follows... [60 60 60 70 70 70 70 80 80 80 80 90 90 90]. When plotted
(Figure XX) using a histogram, it resulted in a symmetric graph which
somewhat resembled a normal curve. The plot of this data has a normal
curve superimposed on it (although this curve does not extend to the
tails of the plot).\newline
\newline
Normality tests in general assume normality and are looking for evidence
to reject that hypothesis. In general small sample sizes or sample
where data point are not all unique are problematic for normality
testing. With very small sample sizes, there needs to be strong
evidence of a deviation from normality in order to get a low p{}-value.
Samples of less than 15 or 20 may produce test results that will
usually not reject the null.\newline
\newline
This data sets analysed here were not suitable for normality testing
using the Anderson{}-Darling (AD) method as this test is only suitable
for sample size over 25. The AD works well for sample sizes between 25
\& 1000 data points[2].\newline
\newline
The Shapiro{}-Wilks test was use for this sample and calculated using
the GNU software R. Using this software, using a sample size outside
the range of 3 to 5000, an error is generated.\newline
\newline
[Results go here in a table]\newline
\newline
These results should not be used to state that your data are
{\textquotedbl}normally distributed{\textquotedbl}. The Shapiro{}-Wilk
test provides evidence of certain types of
{\textquotedbl}non{}-normality{\textquotedbl} and does not guarantee
that data is normal.\newline
\newline
This data is not discrete, i.e. marks were not intentionally given in
multiples of 10, but by using a continuous scale. It is interesting
that although the data increments appear to have steps, the
Shapiro{}-Wilk test did not indicate that the data was not normal.

~ 

\subsection[References ]{\rmfamily \newline
References }
\subsection{~ }
Altman, D. G. (1991). \textit{Practical Statistics for Medical Research}
(First ed.). London: Chapman \& Hall. 

~ 

Dawes, J. (2008). Do Data Characteristics Change According to the number
of scale points used? An experiment using 5{}-point, 7{}-point and
10{}-point scales. \textit{International Journal of Market Research},
50 (1), 61{}-77.

~ 

Minitab available at URL:
\href{http://www.minitab.com/}{http://www.minitab.com}

Moore, D. S., \& McCabe, G. P. (2006). \textit{Introduction to the
Practice of Statistics} (Fifth ed.). New York: W H Freeman and Company.


\newline
~ 

\section[Appendix A: The data file ]{\rmfamily Appendix A: The data file
}

\bigskip

learnerid, totalconcepttime, totalconceptcount,
complete\_initial\_questionaire\_count, correct\_answer\_count,
incomplete\_answer\_count, incorrect\_answer\_count, login\_count,
logout\_count, reset\_profile\_count, view\_concept\_count,
view\_test\_count, elgrid{}-easy, future{}-elgrid,
future{}-use{}-tools, prac{}-apply{}-concepts, prac{}-env{}-easy,
prac{}-good{}-eval, practical{}-access, practical{}-understood,
tests{}-good{}-eval, theory{}-comp{}-prac, too{}-easy, too{}-hard,
traffic{}-light{}-prac, traffic{}-understood, trafficlights{}-nav,
tutor{}-req, projectmark, elgridmark, totalmark, gender, level

2, 2528613, 2518, 3, 118, 2, 18, 199, 151, 14, 2972, 122, ?, ?, ?, ?, ?,
?, ?, ?, ?, ?, ?, ?, ?, ?, ?, ?, ?, ?, ?, F, postgrad

3, 1746, 9, 0, 9, 0, 2, 3, 2, 0, 13, 1, ?, ?, ?, ?, ?, ?, ?, ?, ?, ?, ?,
?, ?, ?, ?, ?, ?, ?, ?, M, postdoc

8, 1246, 24, 1, 10, 0, 2, 8, 4, 0, 46, 6, ?, ?, ?, ?, ?, ?, ?, ?, ?, ?,
?, ?, ?, ?, ?, ?, ?, ?, ?, M, academic

11, 57, 3, 0, 0, 0, 0, 1, 0, 0, 4, 0, ?, ?, ?, ?, ?, ?, ?, ?, ?, ?, ?,
?, ?, ?, ?, ?, ?, ?, ?, M, postdoc

14, 184, 13, 0, 0, 0, 0, 1, 0, 0, 14, 1, ?, ?, ?, ?, ?, ?, ?, ?, ?, ?,
?, ?, ?, ?, ?, ?, ?, ?, ?, M, postdoc

17, 0, 0, 0, 0, 0, 0, 1, 1, 0, 0, 0, ?, ?, ?, ?, ?, ?, ?, ?, ?, ?, ?, ?,
?, ?, ?, ?, ?, ?, ?, F, academic

22, 155, 9, 0, 0, 0, 0, 1, 0, 0, 10, 0, ?, ?, ?, ?, ?, ?, ?, ?, ?, ?, ?,
?, ?, ?, ?, ?, ?, ?, ?, M, postdoc

24, 3636, 35, 0, 0, 0, 0, 5, 0, 0, 42, 0, ?, ?, ?, ?, ?, ?, ?, ?, ?, ?,
?, ?, ?, ?, ?, ?, ?, ?, ?, M, postgrad

29, 3254, 105, 0, 0, 0, 0, 2, 1, 0, 111, 0, ?, ?, ?, ?, ?, ?, ?, ?, ?,
?, ?, ?, ?, ?, ?, ?, ?, ?, ?, M, postdoc

33, 214, 5, 0, 0, 0, 0, 2, 1, 0, 7, 0, ?, ?, ?, ?, ?, ?, ?, ?, ?, ?, ?,
?, ?, ?, ?, ?, ?, ?, ?, M, ?

34, 72530, 194, 1, 4, 0, 2, 10, 7, 0, 224, 3, ?, ?, ?, ?, ?, ?, ?, ?, ?,
?, ?, ?, ?, ?, ?, ?, ?, ?, ?, M, postgrad

39, 4806, 64, 0, 0, 0, 0, 3, 1, 0, 75, 0, Somewhat Agree, ?, ?, Somewhat
Agree, Somewhat Agree, Somewhat Agree, Somewhat Agree, Somewhat Agree,
Somewhat Disagree, Strongly Agree, Somewhat Disagree, Strongly
Disagree, ?, Somewhat Disagree, Somewhat Disagree, Somewhat Agree, ?,
?, ?, M, ?

41, 11094, 109, 0, 0, 0, 0, 8, 5, 0, 134, 1, Somewhat Disagree, ?, ?,
Somewhat Agree, Somewhat Disagree, Somewhat Agree, Somewhat Agree,
Somewhat Disagree, Somewhat Agree, Neither Agree nor Disagree, Somewhat
Disagree, Somewhat Disagree, ?, Somewhat Disagree, Somewhat Agree,
Somewhat Agree, ?, ?, ?, M, ?

43, 10871, 140, 1, 11, 0, 1, 9, 1, 0, 175, 3, ?, ?, ?, ?, ?, ?, ?, ?, ?,
?, ?, ?, ?, ?, ?, ?, ?, ?, ?, M, ?

44, 2849, 50, 0, 0, 0, 0, 10, 9, 0, 73, 1, Somewhat Agree, ?, ?,
Somewhat Agree, Neither Agree nor Disagree, Somewhat Agree, Strongly
Agree, Strongly Agree, Somewhat Agree, Neither Agree nor Disagree,
Somewhat Disagree, Somewhat Disagree, ?, Neither Agree nor Disagree,
Strongly Disagree, Strongly Disagree, ?, ?, ?, M, ?

48, 1419, 24, 0, 0, 0, 0, 1, 0, 0, 27, 0, ?, ?, ?, ?, ?, ?, Neither
Agree nor Disagree, Strongly Agree, ?, ?, Somewhat Disagree, Somewhat
Disagree, ?, ?, ?, Strongly Disagree, ?, ?, ?, M, ?

50, 7808, 187, 0, 0, 0, 0, 8, 7, 0, 218, 0, ?, ?, ?, ?, ?, ?, ?, ?, ?,
?, ?, ?, ?, ?, ?, ?, ?, ?, ?, M, academic

53, 1288, 11, 0, 0, 0, 0, 1, 0, 0, 12, 0, ?, ?, ?, ?, ?, ?, ?, ?, ?, ?,
?, ?, ?, ?, ?, ?, ?, ?, ?, M, ?

54, 11715, 120, 1, 0, 0, 0, 8, 4, 0, 160, 1, Neither Agree nor Disagree,
?, ?, Strongly Agree, Somewhat Agree, Somewhat Agree, Strongly Agree,
Strongly Agree, Somewhat Agree, Neither Agree nor Disagree, Neither
Agree nor Disagree, Strongly Disagree, ?, Somewhat Agree, Somewhat
Agree, Neither Agree nor Disagree, ?, ?, ?, F, ?

55, 6906, 62, 0, 0, 0, 0, 5, 1, 1, 83, 1, Strongly Agree, ?, ?, Strongly
Agree, Strongly Agree, Strongly Agree, Strongly Agree, Strongly Agree,
Somewhat Agree, Strongly Agree, Strongly Disagree, Strongly Disagree,
?, Somewhat Agree, Somewhat Agree, Neither Agree nor Disagree, ?, ?, ?,
M, ?

57, 3191, 15, 0, 0, 0, 0, 4, 3, 0, 24, 1, ?, ?, ?, ?, ?, ?, ?, ?, ?, ?,
?, ?, ?, ?, ?, ?, ?, ?, ?, M, ?

58, 1517, 47, 1, 27, 0, 10, 1, 0, 0, 58, 10, ?, ?, ?, ?, ?, ?, ?, ?, ?,
?, ?, ?, ?, ?, ?, ?, ?, ?, ?, M, ?

59, 0, 0, 0, 0, 0, 0, 2, 2, 0, 0, 0, ?, ?, ?, ?, ?, ?, ?, ?, ?, ?, ?, ?,
?, ?, ?, ?, ?, ?, ?, M, ?

61, 11679, 137, 0, 0, 0, 0, 9, 3, 0, 159, 1, Neither Agree nor Disagree,
?, ?, Somewhat Agree, Somewhat Disagree, Neither Agree nor Disagree,
Somewhat Agree, Somewhat Agree, Somewhat Disagree, Neither Agree nor
Disagree, Somewhat Disagree, Somewhat Disagree, ?, Strongly Disagree,
Strongly Disagree, Somewhat Disagree, ?, ?, ?, M, ?

64, 3774, 51, 0, 0, 0, 0, 8, 5, 0, 73, 0, ?, ?, ?, ?, ?, ?, ?, ?, ?, ?,
?, ?, ?, ?, ?, ?, ?, ?, ?, M, ?

65, 2625, 38, 0, 0, 0, 0, 1, 0, 0, 44, 0, ?, ?, ?, ?, ?, ?, ?, ?, ?, ?,
?, ?, ?, ?, ?, ?, ?, ?, ?, ?, ?

67, 3669, 45, 0, 0, 0, 0, 1, 0, 0, 53, 0, ?, ?, ?, ?, ?, ?, ?, ?, ?, ?,
?, ?, ?, ?, ?, ?, ?, ?, ?, ?, ?

68, 6516, 68, 0, 0, 0, 0, 6, 1, 1, 85, 1, ?, ?, ?, ?, ?, ?, ?, ?, ?, ?,
?, ?, ?, ?, ?, ?, ?, ?, ?, ?, ?

69, 759, 36, 0, 0, 0, 0, 2, 1, 0, 52, 1, ?, ?, ?, ?, ?, ?, ?, ?, ?, ?,
?, ?, ?, ?, ?, ?, ?, ?, ?, ?, ?

70, 6915, 45, 0, 0, 0, 0, 4, 1, 0, 52, 0, ?, ?, ?, ?, ?, ?, ?, ?, ?, ?,
?, ?, ?, ?, ?, ?, ?, ?, ?, ?, ?

71, 8735, 116, 0, 0, 0, 0, 10, 2, 0, 148, 1, ?, ?, ?, ?, ?, ?, ?, ?, ?,
?, ?, ?, ?, ?, ?, ?, ?, ?, ?, ?, ?

72, 3397, 46, 1, 11, 0, 1, 4, 0, 0, 63, 2, ?, ?, ?, ?, ?, ?, ?, ?, ?, ?,
?, ?, ?, ?, ?, ?, ?, ?, ?, ?, ?

73, 2629, 39, 0, 0, 0, 0, 3, 2, 0, 44, 0, ?, ?, ?, ?, ?, ?, ?, ?, ?, ?,
?, ?, ?, ?, ?, ?, ?, ?, ?, ?, ?

74, 8257, 76, 0, 0, 0, 0, 2, 0, 0, 87, 0, ?, ?, ?, ?, ?, ?, ?, ?, ?, ?,
?, ?, ?, ?, ?, ?, ?, ?, ?, ?, ?

75, 6904, 139, 0, 3, 0, 0, 10, 8, 1, 167, 2, Strongly Agree, ?, ?,
Strongly Agree, Strongly Agree, Strongly Agree, Strongly Agree,
Strongly Agree, No answer Given, Strongly Agree, Strongly Disagree,
Strongly Disagree, ?, No answer Given, Somewhat Disagree, Strongly
Disagree, ?, ?, ?, M, postgrad

76, 732600, 377, 1, 0, 0, 0, 28, 0, 0, 457, 3, ?, ?, ?, ?, ?, ?, ?, ?,
?, ?, ?, ?, ?, ?, ?, ?, ?, ?, ?, M, ?

78, 10540, 89, 1, 0, 0, 2, 1, 0, 0, 96, 7, Somewhat Agree, Strongly
Agree, Strongly Agree, Somewhat Agree, Somewhat Agree, No answer Given,
Strongly Agree, Somewhat Agree, No answer Given, No answer Given,
Strongly Disagree, Somewhat Disagree, Somewhat Disagree, Somewhat
Disagree, No answer Given, Strongly Agree, ?, ?, ?, ?, ?

79, 42683, 171, 2, 0, 0, 0, 2, 1, 0, 182, 7, Neither Agree nor Disagree,
Somewhat Agree, Somewhat Agree, Somewhat Agree, Somewhat Agree,
Strongly Agree, Somewhat Agree, Somewhat Agree, Somewhat Agree,
Somewhat Agree, Neither Agree nor Disagree, Strongly Agree, Neither
Agree nor Disagree, Neither Agree nor Disagree, Somewhat Agree,
Somewhat Agree, ?, ?, ?, ?, ?

80, 2256, 34, 0, 0, 0, 0, 3, 2, 1, 39, 2, ?, ?, ?, ?, ?, ?, ?, ?, ?, ?,
?, ?, ?, ?, ?, ?, ?, ?, ?, M, postgrad

81, 10450, 156, 1, 3, 0, 0, 1, 1, 0, 167, 8, Somewhat Agree, Strongly
Agree, Somewhat Agree, Somewhat Agree, Neither Agree nor Disagree,
Somewhat Agree, Somewhat Agree, Somewhat Agree, Strongly Agree,
Somewhat Agree, Strongly Disagree, Strongly Disagree, Somewhat Agree,
Somewhat Disagree, Somewhat Disagree, Strongly Agree, ?, ?, ?, ?, ?

82, 10170, 58, 0, 0, 0, 0, 1, 0, 0, 63, 0, Somewhat Agree, Somewhat
Agree, Strongly Agree, Somewhat Agree, Somewhat Agree, Somewhat Agree,
Somewhat Agree, Somewhat Agree, Somewhat Agree, Somewhat Agree,
Somewhat Disagree, Strongly Disagree, Somewhat Agree, Somewhat Agree,
Somewhat Agree, Somewhat Agree, ?, ?, ?, ?, ?

83, 8895, 48, 0, 0, 0, 0, 1, 0, 0, 51, 0, Somewhat Agree, Somewhat
Disagree, Somewhat Agree, Somewhat Agree, Neither Agree nor Disagree,
No answer Given, Somewhat Agree, Neither Agree nor Disagree, No answer
Given, Neither Agree nor Disagree, Somewhat Disagree, Strongly
Disagree, No answer Given, No answer Given, No answer Given, Somewhat
Agree, ?, ?, ?, ?, ?

84, 10534, 42, 0, 0, 0, 0, 1, 0, 0, 47, 0, Somewhat Agree, Neither Agree
nor Disagree, Somewhat Agree, Strongly Agree, Somewhat Agree, Somewhat
Agree, Neither Agree nor Disagree, Strongly Agree, Neither Agree nor
Disagree, Neither Agree nor Disagree, Strongly Disagree, Somewhat
Agree, Neither Agree nor Disagree, Neither Agree nor Disagree, Somewhat
Agree, Strongly Agree, ?, ?, ?, ?, ?

85, 8774, 55, 0, 0, 0, 0, 1, 0, 0, 60, 1, Somewhat Disagree, Neither
Agree nor Disagree, Strongly Disagree, Neither Agree nor Disagree,
Neither Agree nor Disagree, Neither Agree nor Disagree, Somewhat Agree,
Neither Agree nor Disagree, Neither Agree nor Disagree, Neither Agree
nor Disagree, Neither Agree nor Disagree, Neither Agree nor Disagree,
Neither Agree nor Disagree, Neither Agree nor Disagree, Neither Agree
nor Disagree, Somewhat Agree, ?, ?, ?, ?, ?

102, 275, 9, 0, 0, 0, 0, 1, 0, 0, 10, 0, ?, ?, ?, ?, ?, ?, ?, ?, ?, ?,
?, ?, ?, ?, ?, ?, ?, ?, ?, ?, ?

110, 0, 0, 0, 0, 0, 0, 1, 1, 0, 0, 0, ?, ?, ?, ?, ?, ?, ?, ?, ?, ?, ?,
?, ?, ?, ?, ?, ?, ?, ?, M, ?

111, 0, 0, 0, 0, 0, 0, 1, 1, 0, 0, 0, ?, ?, ?, ?, ?, ?, ?, ?, ?, ?, ?,
?, ?, ?, ?, ?, ?, ?, ?, M, ?

112, 0, 0, 0, 0, 0, 0, 1, 1, 0, 0, 0, ?, ?, ?, ?, ?, ?, ?, ?, ?, ?, ?,
?, ?, ?, ?, ?, ?, ?, ?, M, ?

113, 0, 0, 0, 0, 0, 0, 2, 2, 0, 0, 0, ?, ?, ?, ?, ?, ?, ?, ?, ?, ?, ?,
?, ?, ?, ?, ?, ?, ?, ?, ?, ?

114, 3553, 108, 1, 8, 0, 3, 3, 1, 0, 126, 3, ?, ?, ?, ?, ?, ?, ?, ?, ?,
?, ?, ?, ?, ?, ?, ?, ?, ?, ?, M, ?

115, 0, 0, 0, 0, 0, 0, 2, 1, 0, 1, 0, ?, ?, ?, ?, ?, ?, ?, ?, ?, ?, ?,
?, ?, ?, ?, ?, ?, ?, ?, M, academic

116, 18, 3, 0, 0, 0, 0, 2, 1, 0, 5, 1, ?, ?, ?, ?, ?, ?, ?, ?, ?, ?, ?,
?, ?, ?, ?, ?, ?, ?, ?, M, ?

117, 0, 0, 0, 0, 0, 0, 1, 1, 0, 15, 0, ?, ?, ?, ?, ?, ?, ?, ?, ?, ?, ?,
?, ?, ?, ?, ?, ?, ?, ?, M, academic

118, 3013, 3, 0, 0, 0, 0, 1, 1, 0, 6, 0, ?, ?, ?, ?, ?, ?, ?, ?, ?, ?,
?, ?, ?, ?, ?, ?, ?, ?, ?, ?, ?

124, 0, 0, 0, 0, 0, 0, 1, 0, 0, 0, 0, ?, ?, ?, ?, ?, ?, ?, ?, ?, ?, ?,
?, ?, ?, ?, ?, ?, ?, ?, ?, postgrad

133, 5104, 67, 0, 0, 0, 0, 4, 3, 0, 73, 0, ?, ?, ?, ?, ?, ?, ?, ?, ?, ?,
?, ?, ?, ?, ?, ?, ?, ?, ?, ?, postgrad

135, 14, 1, 0, 0, 0, 0, 1, 1, 0, 4, 0, ?, ?, ?, ?, ?, ?, ?, ?, ?, ?, ?,
?, ?, ?, ?, ?, ?, ?, ?, M, academic

136, 48723, 176, 1, 11, 0, 0, 10, 1, 0, 204, 2, ?, ?, ?, ?, ?, ?, ?, ?,
?, ?, ?, ?, ?, ?, ?, ?, ?, ?, ?, M, postgrad

137, 350, 28, 2, 11, 0, 0, 3, 2, 0, 33, 3, ?, ?, ?, ?, ?, ?, ?, ?, ?, ?,
?, ?, ?, ?, ?, ?, ?, ?, ?, M, postdoc

138, 214, 34, 0, 0, 0, 0, 2, 1, 0, 36, 1, ?, ?, ?, ?, ?, ?, ?, ?, ?, ?,
?, ?, ?, ?, ?, ?, ?, ?, ?, M, ?

149, 3738, 188, 2, 172, 3, 33, 8, 1, 1, 220, 22, Somewhat Agree,
Strongly Disagree, Strongly Disagree, Neither Agree nor Disagree,
Strongly Disagree, Strongly Agree, Somewhat Disagree, Somewhat Agree,
Somewhat Agree, Neither Agree nor Disagree, Neither Agree nor Disagree,
Somewhat Disagree, Somewhat Disagree, Neither Agree nor Disagree,
Neither Agree nor Disagree, Somewhat Agree, 90, 90, 90, M, undergrad

150, 5246, 149, 2, 26, 0, 7, 9, 4, 0, 163, 8, Strongly Agree, Strongly
Agree, Strongly Agree, Strongly Agree, Strongly Agree, Strongly Agree,
Somewhat Agree, Strongly Agree, Strongly Agree, Strongly Agree,
Strongly Disagree, Strongly Disagree, Somewhat Agree, Somewhat Agree,
Strongly Agree, Neither Agree nor Disagree, 70, 60, 66, M, undergrad

151, 1462891, 102, 2, 14, 1, 1, 10, 1, 0, 117, 6, Strongly Agree,
Strongly Agree, Somewhat Agree, Strongly Agree, Somewhat Agree,
Strongly Agree, Somewhat Agree, Somewhat Agree, Somewhat Agree,
Strongly Agree, Strongly Disagree, Strongly Disagree, Strongly Agree,
Strongly Disagree, Strongly Disagree, Strongly Disagree, 90, 100, 94,
M, undergrad

152, 12143, 294, 1, 7, 0, 3, 14, 12, 0, 316, 10, Strongly Agree,
Somewhat Agree, Somewhat Disagree, Somewhat Agree, Neither Agree nor
Disagree, Somewhat Agree, Strongly Agree, Somewhat Agree, Strongly
Agree, Somewhat Agree, Somewhat Disagree, Neither Agree nor Disagree,
Somewhat Disagree, Strongly Agree, Strongly Agree, Strongly Agree, 60,
100, 76, M, undergrad

153, 92493, 198, 1, 24, 1, 2, 13, 2, 0, 218, 9, Strongly Agree, Somewhat
Agree, Neither Agree nor Disagree, Strongly Agree, Neither Agree nor
Disagree, Strongly Agree, Somewhat Agree, Strongly Agree, Strongly
Agree, Strongly Agree, Strongly Disagree, Somewhat Disagree, Neither
Agree nor Disagree, Somewhat Agree, Somewhat Disagree, Somewhat
Disagree, 90, 100, 94, M, undergrad

154, 18597, 201, 3, 8, 0, 2, 11, 4, 0, 224, 13, Somewhat Agree, Neither
Agree nor Disagree, Somewhat Disagree, Neither Agree nor Disagree,
Somewhat Agree, Somewhat Agree, Somewhat Agree, Somewhat Agree,
Strongly Agree, Neither Agree nor Disagree, Somewhat Agree, Strongly
Disagree, Neither Agree nor Disagree, Somewhat Agree, Somewhat Agree,
Somewhat Agree, 70, 100, 82, M, undergrad

155, 12761, 167, 3, 37, 2, 8, 13, 3, 0, 190, 19, Somewhat Agree,
Somewhat Agree, Somewhat Disagree, Somewhat Agree, Neither Agree nor
Disagree, Neither Agree nor Disagree, Somewhat Agree, Somewhat Agree,
Somewhat Disagree, Somewhat Agree, Neither Agree nor Disagree, Strongly
Disagree, Somewhat Disagree, Somewhat Disagree, Somewhat Agree,
Somewhat Agree, 70, 100, 82, M, undergrad

156, 38164, 195, 3, 144, 5, 27, 11, 1, 0, 248, 29, Somewhat Agree,
Somewhat Agree, Somewhat Disagree, Somewhat Agree, Strongly Agree,
Somewhat Agree, Neither Agree nor Disagree, Somewhat Agree, Somewhat
Agree, Somewhat Agree, Somewhat Disagree, Strongly Disagree, Somewhat
Disagree, Somewhat Agree, Somewhat Agree, Strongly Disagree, 80, 90,
84, M, undergrad

157, 23663, 261, 1, 19, 1, 1, 17, 2, 0, 282, 10, Strongly Agree,
Somewhat Agree, Somewhat Disagree, Neither Agree nor Disagree, Somewhat
Disagree, Somewhat Agree, Somewhat Disagree, Strongly Agree, Neither
Agree nor Disagree, Strongly Agree, Strongly Agree, Somewhat Disagree,
Neither Agree nor Disagree, Neither Agree nor Disagree, Neither Agree
nor Disagree, Strongly Agree, 80, 100, 88, M, undergrad

158, 3307, 83, 0, 46, 3, 5, 5, 1, 0, 94, 6, Somewhat Agree, Strongly
Agree, Strongly Disagree, Somewhat Agree, Neither Agree nor Disagree,
Strongly Agree, Somewhat Agree, Strongly Agree, Neither Agree nor
Disagree, Strongly Agree, Neither Agree nor Disagree, Strongly
Disagree, Strongly Agree, Neither Agree nor Disagree, Neither Agree nor
Disagree, Strongly Disagree, 70, 100, 82, M, undergrad

159, 12548, 103, 2, 13, 0, 7, 11, 1, 0, 120, 9, Somewhat Agree, Strongly
Agree, Neither Agree nor Disagree, Somewhat Agree, Somewhat Agree,
Neither Agree nor Disagree, Somewhat Agree, Neither Agree nor Disagree,
Somewhat Agree, Somewhat Agree, Neither Agree nor Disagree, Strongly
Disagree, Strongly Agree, Somewhat Agree, Neither Agree nor Disagree,
Somewhat Disagree, 80, 100, 88, M, undergrad

160, 7368, 137, 1, 13, 0, 7, 8, 1, 0, 151, 9, Somewhat Agree, Somewhat
Agree, Somewhat Agree, Somewhat Agree, Somewhat Agree, Somewhat Agree,
Neither Agree nor Disagree, Somewhat Agree, Strongly Agree, Somewhat
Agree, Somewhat Disagree, Strongly Disagree, Strongly Agree, Strongly
Agree, Strongly Agree, Somewhat Agree, 80, 100, 88, M, undergrad

161, 19072, 565, 7, 188, 3, 33, 21, 3, 5, 626, 72, Neither Agree nor
Disagree, Somewhat Agree, Somewhat Disagree, Somewhat Agree, Neither
Agree nor Disagree, Somewhat Agree, Somewhat Disagree, Neither Agree
nor Disagree, Somewhat Agree, Somewhat Agree, Strongly Disagree,
Neither Agree nor Disagree, Somewhat Agree, Somewhat Agree, Somewhat
Agree, Somewhat Agree, 60, 90, 72, M, undergrad

162, 10767, 123, 1, 19, 0, 3, 9, 3, 0, 135, 6, Somewhat Agree, Strongly
Agree, Strongly Disagree, Neither Agree nor Disagree, Neither Agree nor
Disagree, Neither Agree nor Disagree, Neither Agree nor Disagree,
Neither Agree nor Disagree, Somewhat Disagree, Somewhat Disagree,
Strongly Disagree, Neither Agree nor Disagree, Strongly Disagree,
Somewhat Agree, Somewhat Disagree, Strongly Agree, 60, 100, 76, M,
undergrad

163, 0, 0, 0, 0, 0, 0, 1, 1, 0, 0, 0, ?, ?, ?, ?, ?, ?, ?, ?, ?, ?, ?,
?, ?, ?, ?, ?, ?, ?, ?, F, academic


\bigskip


\bigskip
\end{document}
